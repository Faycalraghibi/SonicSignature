% ==============================================================================
% Faycal Raghibi, Guerouaoui Ilyas — Rapport Bonus : Prédiction Conforme
% IMT Nord Europe — IC2 S2.1 Projet Machine Learning
% ==============================================================================
\documentclass[11pt,a4paper]{article}

% ── Packages ──────────────────────────────────────────────────────────────────
\usepackage[utf8]{inputenc}
\usepackage[T1]{fontenc}
\usepackage[french]{babel}
\usepackage{geometry}
\geometry{margin=2.5cm}
\usepackage{amsmath,amssymb}
\usepackage{graphicx}
\usepackage{booktabs}
\usepackage{hyperref}
\usepackage{enumitem}
\usepackage{float}
\usepackage{caption}
\usepackage{parskip}

\usepackage{xcolor}
\hypersetup{
  colorlinks=true,
  linkcolor=blue!70!black,
  citecolor=blue!70!black,
  urlcolor=blue!70!black
}

% ── Titre ─────────────────────────────────────────────────────────────────────
\title{%
  \textbf{Analyse Musicale Spotify}\\[4pt]
  \large Classification des Genres, Prédiction de Popularité \& Recommandation\\[4pt]
}
\author{Faycal Raghibi, Guerouaoui Ilyas}
\date{Février 2026}

\begin{document}
\maketitle

% ══════════════════════════════════════════════════════════════════════════════
\section{Introduction}
% ══════════════════════════════════════════════════════════════════════════════

Les plateformes de streaming musical telles que Spotify hébergent des millions
de titres couvrant une grande variété de genres.  La classification automatique
des genres, la prédiction de popularité et la recommandation basée sur le
contenu sont des tâches fondamentales qui améliorent l'expérience utilisateur
et la curation de la plateforme.  Ce rapport présente la méthodologie et les
résultats d'un pipeline d'apprentissage automatique construit autour d'un jeu
de données Spotify contenant des descripteurs audio numériques extraits de
l'API Web Spotify.

Le projet aborde trois objectifs interconnectés :
\begin{enumerate}[leftmargin=*]
  \item \textbf{Classification des genres} — attribuer l'une des étiquettes de
        genre à chaque morceau en fonction de ses caractéristiques audio.
  \item \textbf{Prédiction de popularité} — estimer le score continu de
        popularité d'un morceau par régression.
  \item \textbf{Recommandation basée sur le contenu} — suggérer des morceaux
        similaires en mesurant la proximité dans l'espace des caractéristiques.
\end{enumerate}

Toutes les expériences ont été implémentées en Python à l'aide de Scikit-Learn,
NumPy, Pandas, Matplotlib et Seaborn.  La suite de ce rapport est organisée
comme suit : la Section~\ref{sec:data} décrit les jeux de données ; la
Section~\ref{sec:preproc} détaille le pipeline de prétraitement ; les
Sections~\ref{sec:classif} et~\ref{sec:pop} présentent respectivement les
tâches de classification et de régression ; la Section~\ref{sec:reco}
introduit l'approche de recommandation ; la Section~\ref{sec:bonus} décrit
l'expérience bonus de prédiction conforme ; et la Section~\ref{sec:concl}
conclut par une discussion.

% ══════════════════════════════════════════════════════════════════════════════
\section{Description des Données \& Analyse Exploratoire}
\label{sec:data}
% ══════════════════════════════════════════════════════════════════════════════

\subsection{Jeux de Données Disponibles}

Quatre fichiers CSV ont été fournis :

\begin{table}[H]
\centering
\small
\begin{tabular}{@{}llll@{}}
\toprule
\textbf{Fichier} & \textbf{Lignes} & \textbf{Colonnes} & \textbf{Objectif} \\
\midrule
\texttt{spotify\_dataset\_train.csv}  & 25\,493 & 17 & Entraînement (incl.\ \texttt{genre}) \\
\texttt{spotify\_dataset\_test.csv}   & 2\,834  & 16 & Test (sans étiquette \texttt{genre}) \\
\texttt{spotify\_dataset\_subset.csv} & —       & —  & Sous-ensemble pour la régression de popularité \\
\texttt{recommendation\_spotify.csv}  & —       & —  & Pool pour le système de recommandation \\
\bottomrule
\end{tabular}
\caption{Aperçu des jeux de données du projet.}
\label{tab:datasets}
\end{table}

\subsection{Dictionnaire des Caractéristiques}

Chaque morceau est décrit par les caractéristiques listées dans le
Tableau~\ref{tab:features}.

\begin{table}[H]
\centering
\small
\begin{tabular}{@{}lll@{}}
\toprule
\textbf{Caractéristique} & \textbf{Type} & \textbf{Description} \\
\midrule
\texttt{track\_id}        & chaîne  & Identifiant unique Spotify \\
\texttt{track\_name}      & chaîne  & Nom du morceau \\
\texttt{artists}          & chaîne  & Artiste(s) interprète(s) \\
\texttt{release\_date}    & chaîne  & Date de sortie (année extraite) \\
\texttt{acousticness}     & flottant & Mesure de confiance de la qualité acoustique \\
\texttt{danceability}     & flottant & Aptitude à la danse \\
\texttt{duration\_ms}     & entier   & Durée du morceau en millisecondes \\
\texttt{energy}           & flottant & Mesure d'intensité perceptive \\
\texttt{instrumentalness} & flottant & Probabilité que le morceau soit instrumental \\
\texttt{key}              & entier   & Tonalité musicale (0–11, classe de hauteur) \\
\texttt{liveness}         & flottant & Probabilité de performance en direct \\
\texttt{loudness}         & flottant & Intensité sonore globale en dB \\
\texttt{mode}             & entier   & Modalité (0 = mineur, 1 = majeur) \\
\texttt{speechiness}      & flottant & Présence de paroles \\
\texttt{tempo}            & flottant & BPM estimé \\
\texttt{valence}          & flottant & Positivité musicale \\
\texttt{popularity}       & entier   & Score de popularité (0–100) \\
\texttt{explicit}         & entier   & Indicateur de contenu explicite \\
\texttt{genre}            & chaîne  & Étiquette de genre (jeu d'entraînement uniquement) \\
\bottomrule
\end{tabular}
\caption{Dictionnaire des caractéristiques des jeux de données Spotify.}
\label{tab:features}
\end{table}

\subsection{Observations Exploratoires}

Une exploration initiale du jeu d'entraînement a révélé plusieurs
caractéristiques qui ont influencé la conception du pipeline :

\begin{itemize}[leftmargin=*]
  \item \textbf{Déséquilibre des classes} — La distribution des genres est
        très inégale.  Certains genres (par ex.\ \textit{pop}, \textit{rock})
        dominent le jeu de données tandis que d'autres sont significativement
        sous-représentés.  Cela a motivé l'utilisation de poids équilibrés
        par classe dans le classificateur.
  \item \textbf{Valeurs manquantes} — Une faible proportion d'entrées
        contient des valeurs numériques manquantes, nécessitant une imputation.
  \item \textbf{Échelles des caractéristiques} — Les caractéristiques couvrent
        des plages très différentes (par ex.\ \texttt{duration\_ms}
        $\sim10^5$ vs.\ \texttt{acousticness} $\in[0,1]$), rendant la
        standardisation nécessaire.
  \item \textbf{Dimensionnalité} — Une visualisation ACP des caractéristiques
        standardisées a montré un chevauchement substantiel entre les genres
        dans les deux premières composantes principales, laissant présager
        une tâche de classification difficile.
\end{itemize}

% ══════════════════════════════════════════════════════════════════════════════
\section{Pipeline de Prétraitement}
\label{sec:preproc}
% ══════════════════════════════════════════════════════════════════════════════

Les étapes de prétraitement suivantes ont été appliquées de manière cohérente
aux données d'entraînement et de test :

\begin{enumerate}[leftmargin=*]
  \item \textbf{Extraction de l'année.}  La colonne \texttt{release\_date} a
        été analysée pour extraire une caractéristique numérique
        \texttt{year}, fournissant un signal temporel compact.
  \item \textbf{Sélection des colonnes.}  Les identifiants non prédictifs
        (\texttt{track\_id}, \texttt{track\_name}, \texttt{artists}) et la
        chaîne brute \texttt{release\_date} ont été supprimés.
  \item \textbf{Imputation des valeurs manquantes.}  Les valeurs manquantes
        restantes dans les colonnes numériques ont été remplies à l'aide de
        \texttt{SimpleImputer} avec une stratégie par la \emph{moyenne}.
  \item \textbf{Standardisation des caractéristiques.}  Toutes les
        caractéristiques numériques ont été centrées et mises à l'échelle
        à variance unitaire via \texttt{StandardScaler}, ajusté sur le jeu
        d'entraînement et appliqué aux deux sous-ensembles.
  \item \textbf{Encodage catégoriel.}  La caractéristique catégorielle binaire
        \texttt{explicit} a été conservée telle quelle (déjà encodée en 0/1).
        Un encodage one-hot a été appliqué si nécessaire pour tout signal
        catégoriel supplémentaire.
\end{enumerate}

Après le prétraitement, la matrice de caractéristiques contenait les colonnes
suivantes utilisées pour la modélisation : \texttt{acousticness},
\texttt{danceability}, \texttt{duration\_ms}, \texttt{energy},
\texttt{instrumentalness}, \texttt{key}, \texttt{liveness}, \texttt{loudness},
\texttt{mode}, \texttt{speechiness}, \texttt{tempo}, \texttt{valence},
\texttt{popularity}, \texttt{explicit} et \texttt{year}.

% ══════════════════════════════════════════════════════════════════════════════
\section{Classification des Genres}
\label{sec:classif}
% ══════════════════════════════════════════════════════════════════════════════

\subsection{Sélection du Modèle}

Un \textbf{Classificateur Random Forest} a été choisi pour la prédiction
multi-classe des genres.  Les forêts aléatoires agrègent de nombreux arbres de
décision décorrélés via l'agrégation bootstrap, offrant une robustesse contre
le surapprentissage et la capacité de gérer des types de caractéristiques
mixtes.  Les hyperparamètres suivants ont été définis :

\begin{itemize}[leftmargin=*]
  \item \texttt{n\_estimators} = 100 — nombre d'arbres dans l'ensemble.
  \item \texttt{class\_weight} = \texttt{balanced} — ajuste automatiquement
        les poids des échantillons de manière inversement proportionnelle aux
        fréquences des classes, atténuant l'effet du déséquilibre des classes.
  \item \texttt{random\_state} = 42 — graine fixe pour la reproductibilité.
\end{itemize}

\subsection{Protocole d'Évaluation}

Le modèle a été évalué par \textbf{validation croisée stratifiée à 5 plis}
sur le jeu d'entraînement.  La métrique principale est le \textbf{score F1
micro-moyenné}, qui calcule la précision et le rappel globaux sur toutes les
classes et est équivalent à l'exactitude lorsque chaque échantillon se voit
attribuer exactement une étiquette.

\subsection{Résultats}

\begin{table}[H]
\centering
\begin{tabular}{@{}lc@{}}
\toprule
\textbf{Métrique} & \textbf{Valeur} \\
\midrule
F1 CV (micro) moyenne            & 0.4396 \\
F1 CV (micro) écart-type         & $\pm\,0.0132$ \\
\bottomrule
\end{tabular}
\caption{Résultats de la validation croisée pour la classification des genres.}
\label{tab:classif}
\end{table}

Un F1 micro-moyenné d'environ \textbf{0.44} reflète la difficulté inhérente
de la tâche : de nombreux genres partagent des profils audio similaires, et
l'espace des caractéristiques (descripteurs audio purement numériques) ne
capture pas nécessairement les différences stylistiques de haut niveau.  La
pondération équilibrée des classes permet de s'assurer que les genres
minoritaires ne sont pas systématiquement ignorés, mais la séparabilité
globale reste limitée.

\subsection{Prédiction sur le Jeu de Test}

Après la validation croisée, le classificateur a été ré-entraîné sur
l'ensemble du jeu d'entraînement et utilisé pour prédire les étiquettes de
genre des 2\,834 morceaux non étiquetés du jeu de test.  Les prédictions ont
été exportées dans \texttt{submission.csv} au format requis
(\texttt{track\_id}, \texttt{genre}).

% ══════════════════════════════════════════════════════════════════════════════
\section{Prédiction de Popularité}
\label{sec:pop}
% ══════════════════════════════════════════════════════════════════════════════

\subsection{Définition de la Tâche}

La colonne \texttt{popularity} (entier dans $[0,100]$) sert de cible de
régression.  Cette tâche utilise le fichier dédié
\texttt{spotify\_dataset\_subset.csv}, qui se concentre sur une sélection
de morceaux.

\subsection{Modèle}

Un \textbf{Régresseur Random Forest} a été utilisé, reprenant la philosophie
d'ensemble employée pour la classification.  Les régresseurs à base d'arbres
de décision sont bien adaptés pour capturer les relations non linéaires entre
les caractéristiques audio et la popularité sans nécessiter d'ingénierie de
caractéristiques explicite.

\subsection{Résultats}

\begin{table}[H]
\centering
\begin{tabular}{@{}lc@{}}
\toprule
\textbf{Métrique} & \textbf{Valeur} \\
\midrule
Erreur Quadratique Moyenne (MSE) & 517.99 \\
Coefficient de Détermination ($R^2$) & 0.2126 \\
\bottomrule
\end{tabular}
\caption{Performance de la régression de popularité.}
\label{tab:pop}
\end{table}

Un $R^2$ d'environ \textbf{0.21} indique que les caractéristiques audio
seules expliquent approximativement 21\% de la variance de la popularité.
Ceci est attendu car la popularité est fortement influencée par des facteurs
externes — notoriété de l'artiste, marketing, placement dans les playlists,
tendances temporelles — qui ne sont pas capturés par les descripteurs audio.
Le MSE de 517.99 correspond à une erreur quadratique moyenne d'environ
22.8 points de popularité sur l'échelle de 0 à 100.

% ══════════════════════════════════════════════════════════════════════════════
\section{Recommandation Basée sur le Contenu}
\label{sec:reco}
% ══════════════════════════════════════════════════════════════════════════════

\subsection{Approche}

Une stratégie de \textbf{filtrage basé sur le contenu} a été implémentée en
utilisant la \textbf{similarité cosinus} sur les vecteurs de caractéristiques
audio standardisés.  Étant donné un morceau requête, le système récupère les
$k$ morceaux les plus similaires du pool
\texttt{recommendation\_spotify.csv} en classant les similarités cosinus
par paires.

\subsection{Pipeline}

\begin{enumerate}[leftmargin=*]
  \item Charger et prétraiter le jeu de données de recommandation (même
        pipeline que la Section~\ref{sec:preproc}).
  \item Calculer la matrice de similarité cosinus entre toutes les paires de
        morceaux.
  \item Pour un morceau requête donné, trier les scores de similarité par
        ordre décroissant et retourner les $k$ plus proches voisins (en
        excluant le morceau requête lui-même).
\end{enumerate}

La similarité cosinus est un choix naturel dans ce contexte car elle mesure
la proximité angulaire entre les vecteurs de caractéristiques, la rendant
invariante aux changements d'échelle uniformes — une propriété importante
lorsque les caractéristiques ont été standardisées mais peuvent encore
présenter des plages dynamiques différentes selon le sous-ensemble.

% ══════════════════════════════════════════════════════════════════════════════
\section{Bonus : Prédiction Conforme}
\label{sec:bonus}
% ══════════════════════════════════════════════════════════════════════════════

En tant qu'expérience supplémentaire, un cadre de \textbf{prédiction conforme}
a été appliqué pour quantifier l'incertitude de prédiction sur la tâche de
classification des genres.

\subsection{Méthode}

\begin{enumerate}[leftmargin=*]
  \item Un classificateur \textbf{K plus proches voisins (KNN)} a été entraîné
        sur le jeu d'entraînement après prétraitement standard.
  \item Un \textbf{jeu de calibration} de 1\,500 échantillons a été réservé à
        partir des données d'entraînement pour calculer les scores de
        non-conformité.
  \item Pour chaque instance de test, le prédicteur conforme produit un
        \textbf{ensemble de prédiction} — un sous-ensemble d'étiquettes de
        genre garanti de contenir la vraie étiquette avec une probabilité
        d'au moins $1-\alpha$, où $\alpha$ est le niveau de signification
        spécifié par l'utilisateur.
\end{enumerate}

\subsection{Résultats}

\begin{table}[H]
\centering
\begin{tabular}{@{}cccc@{}}
\toprule
$\alpha$ & \textbf{Couverture (\%)} & \textbf{Taille moy.\ de l'ensemble} & \textbf{Ensembles vides (\%)} \\
\midrule
0.01 & 99.88 & 16.74 & 0.00 \\
0.05 & 98.32 & 13.96 & 0.00 \\
0.10 & 96.93 & 12.36 & 0.00 \\
0.20 & 93.82 & 10.19 & 0.00 \\
\bottomrule
\end{tabular}
\caption{Résultats de la prédiction conforme à différents niveaux de signification.}
\label{tab:conformal}
\end{table}

Le prédicteur conforme atteint les garanties de couverture souhaitées à chaque
niveau de signification.  Cependant, les tailles moyennes des ensembles de
prédiction sont importantes (par ex.\ $\approx14$ genres à $\alpha=0.05$),
reflétant le pouvoir discriminant limité des caractéristiques audio pour une
séparation fine des genres.  Cela corrobore le score F1 modeste observé dans
la Section~\ref{sec:classif}.

% ══════════════════════════════════════════════════════════════════════════════
\section{Discussion \& Conclusion}
\label{sec:concl}
% ══════════════════════════════════════════════════════════════════════════════

Ce projet a démontré un pipeline d'apprentissage automatique de bout en bout
pour l'analyse de morceaux Spotify, couvrant la classification, la régression
et la recommandation.  Plusieurs observations méritent discussion :

\begin{itemize}[leftmargin=*]
  \item \textbf{Limitations des caractéristiques.}  Les caractéristiques audio
        purement numériques fournies par l'API Spotify capturent des propriétés
        timbrales, rythmiques et harmoniques de bas niveau mais ne peuvent pas
        encoder des attributs sémantiques ou culturels de plus haut niveau.
        L'incorporation de métadonnées textuelles (paroles, biographie de
        l'artiste) ou de représentations spectrales (par ex.\ spectrogrammes
        Mel) pourrait améliorer significativement la classification des genres.
  \item \textbf{Déséquilibre des classes.}  Malgré l'utilisation de poids
        équilibrés par classe, les genres sous-représentés restent difficiles
        à classifier.  Des techniques telles que le suréchantillonnage (SMOTE)
        ou la classification hiérarchique (regroupement de genres apparentés)
        pourraient apporter des améliorations.
  \item \textbf{Modélisation de la popularité.}  Un $R^2$ de 0.21 confirme que
        la popularité n'est que faiblement liée aux propriétés audio
        intrinsèques.  Un ensemble de caractéristiques plus riche incluant des
        métriques de réseaux sociaux, le calendrier de sortie et l'exposition
        dans les playlists serait nécessaire pour une prévision pratique de la
        popularité.
  \item \textbf{Recommandation.}  Le système de recommandation basé sur la
        similarité cosinus fournit une référence simple mais efficace.  Des
        approches hybrides combinant filtrage basé sur le contenu et filtrage
        collaboratif pourraient offrir des recommandations plus diversifiées
        et précises.
  \item \textbf{Quantification de l'incertitude.}  L'expérience de prédiction
        conforme illustre que les garanties de couverture statistique se font
        au prix d'ensembles de prédiction volumineux lorsque le classificateur
        sous-jacent a une précision modérée.  Améliorer le classificateur de
        base resserrerait directement les intervalles conformes.
\end{itemize}

Dans l'ensemble, la famille Random Forest s'est avérée être un choix fiable et
interprétable pour la classification et la régression sur des caractéristiques
audio tabulaires.  Les travaux futurs pourraient explorer les arbres à gradient
boosté (XGBoost, LightGBM), les embeddings neuronaux ou la fusion multimodale
pour améliorer davantage les performances.

\end{document}
